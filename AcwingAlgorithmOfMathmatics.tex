\documentclass[]{book}
\usepackage{ctex}
\usepackage[hidelinks]{hyperref}
\usepackage{listings}
\usepackage{xcolor}
\definecolor{commentgreen}{RGB}{2,112,10}
\definecolor{eminence}{RGB}{108,48,130}
\definecolor{weborange}{RGB}{255,165,0}
\definecolor{frenchplum}{RGB}{129,20,83}


\title{AcWing算法基础、提高课 数学知识}
\author{廖涛}
\date{2022年4月15日}


\begin{document}
\lstset{
    language=C++,
    frame=tb,
    tabsize=4,
    showstringspaces=false,
    numbers=left,
    %upquote=true,
    commentstyle=\color{commentgreen},
    keywordstyle=\color{eminence},
    stringstyle=\color{red},
    basicstyle=\small\ttfamily, % basic font setting
    emph={int,char,double,float,unsigned,void,bool},
    emphstyle={\color{blue}},
    % escapechar=\&,
    % keyword highlighting
    classoffset=1, % starting new class
    otherkeywords={>,<,.,;,-,!,=,~},
    morekeywords={>,<,.,;,-,!,=,~},
    keywordstyle=\color{weborange},
    classoffset=0,
}


\maketitle
\tableofcontents

\chapter{质数}
$\forall n\in [(1, +\infty) \cap Z^+]$,若 n 只有 1 和 n 两个约数,
那么 n 为质数(素数),否则为合数.

0、1以及负整数既不是质数也不是合数.

\section{判定质数}
\begin{lstlisting}
bool is_prime(int n)
{
    if(n < 2) return false;

    for(int i = 2; i <= n / i; i ++)
    {
        if(n % i == 0) return false;
    }
    return true;
}
\end{lstlisting}

1.是否是大于等于2的整数,2. 从 2~n-1 中是否有 n 的约数.

不要写$i*i<n$,可能 $i*i$结果溢出 int;也不要在循环条件里写$i<=sqrt(n)$,
否则每次循环都会调用$sqrt()$.

$d \mid n$,则必有$\frac{n}{d} \mid n$,故不需要枚举到 $n - 1$,即枚举每组
约数的最小值即可.

时间复杂度为$O(\sqrt{n})$

\section{分解质因数}
\begin{lstlisting}
void divide(int n)
{
    for(int i = 2; i <= n / i; i ++)
    {   
        int cnt = 0;
        if(n % i == 0)
        {
            while(n % i == 0)
            {
                n /= i;
                cnt ++;
            }
            printf("%d %d\n", i, cnt);
        }
        
    }
    
    if(n > 1) printf("%d 1\n", n);
    puts("");
}
\end{lstlisting}

从 2 开始枚举到$\sqrt{n}$,每找到一个质因子则将 n 中的该质因子除尽,最后判断n是否
大于 1,若大于1,则此时的 n 也是所求质因子之一.

Tips1:所有合数因子在被枚举到之前会被更小的质数因子除尽.故枚举到的合数必定不会满足$n \bmod i = 0$.
eg:$n = 24$,8 这个合数因子必然会被之前出现的 2 除尽(除3次).

Tips2:n 最多只存在 1 个大于等于$\sqrt{n}$的质因子,故只需要枚举到$\sqrt{n}$再
判断最后的 n 情况如何即可.简单证明,假设存在
\[ a \mid n, b \mid n, a>\sqrt{n}, b>\sqrt{n}\]
前两个条件可得$a \times b \mid n$,最后一个条件可得$a \times b > n$,相悖,故假设不成立,故 n 最多只存在
1 个大于等于$ \sqrt{n}$ 的质因子.

Tips3:当 $n = 2^k$ 时,第一次枚举即能把 n 除尽,此时的时间复杂度为 $\log_{2}{n}$,
最坏情况则是需要枚举到$\sqrt{n}$,故该方法的时间复杂度为为 $O(\log_{2}{n})$与$O(\sqrt{n})$之间.
\section{质数筛}
\subsection{埃氏筛法}
\begin{lstlisting}
void get_primes(int n)
{
    for(int i = 2; i <= n; i ++)
    {
        if(st[i]) continue;
        
        primes[cnt ++] = i;
        for(int j = 2 * i; j <= n; j += i)
        {
            st[j] = true;
        }
    }
}
\end{lstlisting}

从 2 枚举到 n,即能找到 n 以内的所有质数.每次枚举到的 i 若 st[i] 为 false 则说明 i 为质数,记录到 primes[]中,
并筛掉其倍数.否则 i 为合数,直接跳过本次循环.

这里说明为什么不需要用合数筛掉其他数,当 st[i]为 true 时证明之前已经找到了 i 的一个或多个质因子,
i 的倍数也必定是这些质因子的倍数,所以当 i 为合数时,i 的倍数已经在之前被其质因子筛掉,不需要再筛.

在枚举到 i 时,$2 ~ i-1$的所有数都已经枚举过,若 st[i] 为 false 则说明 $2 ~ i-1$均不是 i 的因数,则 i 为质数(素数),否则为合数.

时间复杂度分析,当$i=2$时内层循环次数为$\frac{n}{2}$,当$i=3$时为$\frac{n}{3}$...,这里我们假设合数也进行内层循环,则内层循环的次数和为
\[\frac{n}{2} + \frac{n}{3} + \frac{n}{4} + \cdots + \frac{n}{n} = n \times (\frac{1}{2} + \frac{1}{3} + \frac{1}{4} + \cdots + \frac{1}{n})\]
调和级数
\[\lim_{n \to \infty} \frac{1}{2} + \frac{1}{3} + \frac{1}{4} + \cdots + \frac{1}{n} = \ln_{}{n}\]
故内层循环的次数小于等于$n\ln_{}{n}$小于$nlog_{2}{n}$,故可以认为时间复杂度为
\[O(n\log_{2}{n})\]
只用质数筛选的情况(埃氏筛法)的时间复杂度为
\[O(n\log_{2}{\log_{2}{n}})\]
可以近似认为是$O(N)$

\subsection{欧拉筛法(线形筛法)}

\begin{lstlisting}
void get_primes(int n)
{
    for(int i = 2; i <= n; i ++)
    {
        if(!st[i]) primes[cnt ++] = i;
        
        for(int j = 0; primes[j] <= n / i; j ++)
        {
            st[primes[j] * i] = true;
            if(i % primes[j] == 0) break;
        }
    }
}
\end{lstlisting}

欧拉筛法的一个特点之一就是每个合数都是用其最小的质因子筛掉.

这里重点讲内层循环.首先无论 i 是否为质数,都会进内层循环.$primes[]$ 中小的数一定在前面.

当 $i \bmod primes[j] != 0$ 时,$primes[j]$ 的不是 i 的因子,则$primes[j] * i$的最小质因子就是$primes[j]$.

当 $i \bmod primes[j] == 0$ 时, $primes[j]$ 是 i 的最小质因子,也是$primes[j] * i$的最小质因子.但当 j ++ 后,
$primes[j + 1] * i$ 有更小的质因子 $primes[j]$($primes[j] \mid i$)这里就不满足用最小质因子筛去合数的要求,所以要break.

每个合数$i$在被枚举到之前一定会被筛掉。假设合数$i$的最小质因子为$pj$,另一个因子$i/pj$必然会在$i$之前被枚举到,当$i/pj$被枚举到时,内层循环
则会把合数$i$筛掉。
\chapter{约数}
\section{试除法求约数}
\begin{lstlisting}
vector<int> get_divisors(int x)
{
vector<int> res;
for(int i = 1; i <= x / i; i ++)
{
    if(x % i == 0)
    {
        res.push_back(i);
        if(i * i != x) res.push_back(x / i);
    }
}
sort(res.begin(), res.end());
return res;
}
\end{lstlisting}
\section{约数个数}
对于整数n, 由算数基本定理
\[n = p_{1}^{a_{1}} * p_{2}^{a_{2}} * p_{3}^{a_{3}} *\cdots* p_{n}^{a_{n}}\]
那么 n 的约数个数为
\[(a_{1} + 1) * (a_{2} + 1) * (a_{3} + 1) *\cdots* (a_{n} + 1)\]
简单证明,对于$n$的每个约数$d$都是这样的形式
\[d = p_{1}^{b_{1}} * p_{2}^{b_{2}} *\cdots* p_{n}^{b_{n}}, 0 \leq b_{i} \leq a_{i}\]
每一个$b_{i}$的不同取值即取得了一个不同的$d$, 故$b_{i}$不同的组合数量就是所求的约数个数.
\begin{lstlisting}
#include <iostream>
#include <algorithm>
#include <unordered_map>

using namespace std;

const int N = 110, mod = 1e9 + 7;
typedef long long LL;

int main()
{
    int n;
    cin >> n;
    unordered_map<int, int> primes;
    
    while(n --)
    {
        int x;
        cin >> x;
        for(int i = 2; i <= x / i; i ++)
        {
            while(x % i == 0)
            {
                x /= i;
                primes[i] ++;
            }
        }
        if(x > 1) primes[x] ++;
    }
    
    LL res = 1;
    for(auto t : primes)
    {
        LL p = t.first, s = t.second;
        res = res * (s + 1) % mod;
    }
    
    cout << res << endl;
    
    return 0;
}
\end{lstlisting}
\section{约数之和}
对于整数n, 由算数基本定理
\[n = p_{1}^{a_{1}} * p_{2}^{a_{2}} * p_{3}^{a_{3}} *\cdots* p_{n}^{a_{n}}\]
那么 n 的约数之和为
\[(p_{1}^{0} + p_{1}^{1} + \cdots + p_{1}^{a_{1}}) * (p_{2}^{0} + p_{2}^{1} + \cdots + p_{2}^{a_{2}}) * (p_{3}^{0} + p_{3}^{1} + \cdots + p_{3}^{a_{3}}) *\cdots*(p_{n}^{0} + p_{n}^{1} + \cdots + p_{n}^{a_{n}})\]
简单理解为什么这样是所有约数之和,用乘法分配律对上式展开,相当于每个括号内选一项出来相乘,乘积相加。容易理解每个括号内选出来的一项相乘得到的结果就是一个约数,再相加就是所有约数之和。

\begin{lstlisting}
#include <iostream>
#include <algorithm>
#include <unordered_map>

using namespace std;

typedef long long LL;
const int mod = 1e9 + 7;
unordered_map<int, int> primes;
int n;

//分解质因数
void divide(int x)
{
    for(int i = 2; i <= x / i; i ++)
    {
        if(x % i == 0)
        {
            while(x % i == 0)
            {
                primes[i] ++;//质因数 i 的次数 +1
                x /= i;
            }
        }
    }
    if(x > 1) primes[x] ++;
}

int main()
{
    cin >> n;
    while(n --)
    {
        int x;
        scanf("%d", &x);
        divide(x);
    }
    
    LL res = 1;
    for(auto prime : primes)
    {
        int s = prime.second, nums = prime.first;
        LL tmp = 1;
        while(s --) tmp = (tmp * nums + 1) % mod;
        res = res * tmp % mod;
    }
    cout << res << endl;
    
    return 0;
}
\end{lstlisting}
这里是直接模拟了公式中的运算从头到尾直接循环相加,相乘,但是当分解出的质数个数较多时,这样的运算速度不够快。
观察公式可以看到每个括号里的序列都是一个等比数列,可以直接利用等比数列的求和公式来计算。等比数列求和公式中的除法
运算转换为乘以其逆元即可,因为这里的模数是质数,所以可以直接用快速幂求逆元。
\begin{lstlisting}
#include <iostream>
#include <unordered_map>
#include <algorithm>
using namespace std;
typedef long long LL;
const int N = 2e9 + 10, mod = 1e9 + 7;

unordered_map<int, int> primes;

void get_primes(int n)
{
    for(int i = 2; i <= n / i; i ++)
    {
        if(n % i == 0)
        {
            while(n % i == 0)
            {
                primes[i] ++;
                n /= i;
            }
        }
    }
    if(n > 1) primes[n] ++;
}

LL qmi(LL a, LL b, LL p)
{
    LL res = 1;
    while(b)
    {
        if(b & 1) res = res * a % p;
        a = a * a % p;
        b >>= 1;
    }
    return res;
}

int main()
{
    int T;
    cin >> T;
    while(T --)
    {
        int a;
        cin >> a;
        get_primes(a);
    }
    
    LL res = 1;
    for(auto e : primes)
    {
        int p = e.first, k = e.second;  
        if((p - 1) % mod == 0)
        {
            res = res * (k + 1) % mod;
        }
        else
        {
            res = res * (qmi(p, k + 1, mod) - 1) % mod * qmi(p - 1, mod - 2, mod) % mod;
        }
    }
    cout << (res % mod + mod) % mod << endl;
    return 0;
}
\end{lstlisting}
注意是从$p^0$一直加到$p^k$,即首项是$p^0=1$,共有$k+1$项。求逆元的时候要判断逆元是否存在,若逆元不存即为$(p-1)\bmod mod = 0$,即$p\bmod mod = 1$,这种情况直接让$res$
乘以$(k+1) * 1$即可。\\
补充等比数列的求和公式,若 
\[S = a_{1} + a_{2} + a_{3}+\cdots+a_{n}\]
并且
\[\forall i \in [1, n], p = \frac{a_{i+1}}{a_{i}}= \frac{a_{i+2}}{a_{i+1}}, p\neq 1\]
则有
\[S = \frac{a_{1}*(p^{n}-1)}{p-1}\]
\section{最大公约数}
$(a / \gcd(a, b)) * (b / \gcd(a, b))* \gcd(a, b)$ 即$a / \gcd(a, b) * b$为 a和b 的最小公倍数.
\begin{lstlisting}
int gcd(int a, int b)
{
    return b ? gcd(b, a % b) : a;
}
\end{lstlisting}
若
\[d \mid a, d \mid b\]
那么有
\[d \mid ax + by\]
即$d$也能整除$a$和$b$的线性组合

\[a \bmod b = a - \left\lfloor \frac{a}{b}\right\rfloor * b\]
令$c = \left\lfloor \frac{a}{b}\right\rfloor$则有
\[a \bmod b = a - c * b\]
即$a\bmod b$也是$a$和$b$的线形组合,故当$d\mid a, d\mid b$时, $d \mid a \bmod b$也成立。\\
而当$d\mid b, d\mid a\bmod b$时候,$a\bmod b = a - c * b$和$b$的一个线性组合$a - c*b + c * b = a$也能被 $d$整除,即$d\mid a$。

这样就说明了$a$和$b$的约数也是$b$和$a\bmod b$的约数,反过来$b$和$a\bmod b$的约数也是$a$的约数。

当$b=0$时,$a$和$b$最大公约数就为$a$, $a \mid a$且$a \mid 0$。
\chapter{欧拉函数}
n的欧拉函数为$[1,\quad n - 1]$内与 n 互质的数的个数.\\
若 
\[n = p_{1}^{a_{1}} + p_{2}^{a_{2}} + p_{3}^{a_{3}} +\cdots + p_{n}^{a_{n}}\]
那么
\[\phi(n) = n * (1 - \frac{1}{p_{1}}) * (1 - \frac{1}{p_{2}}) *\cdots* (1 - \frac{1}{p_{n}}) = 
n*(\frac{p_{1} - 1}{p_{1}})*(\frac{p_{2} - 1}{p_{2}})*\cdots *(\frac{p_{n} - 1}{p_{n}})\]
当 n 为质数时
\[\phi(n) = n - 1\]

\section{费马小定理}
若 a 为整数,p 为质数,那么
\[a^{\phi(p)}\equiv 1 \pmod{p}\]\\即
\[a^{p-1}\equiv 1 \pmod{p}\]
\chapter{快速幂}
\section{快速幂原理}
\begin{lstlisting}
LL qmi(LL a, LL b, LL p)
{
LL res = 1;
while(b)
{
    if(b & 1) res = res * a % p;
    a = a * a % p;
    b >>= 1;
}
return res;
}
\end{lstlisting}
从低到高枚举指数的每一二进制位,若该位为 1 则结果乘以一次 a,同时每次枚举每一位时 a 都要 $a = a * a$.每次直接将指数 b 的最后一位右移去掉.
eg: 当判断第一位是否为 1 时,$a = a^{1}$; 第二位是否为 1 时, $a = a^{2}$, ..., 第n位是否为 1 时, $a = a^{n}$. (类似 a 进制).

\section{快速幂求逆元}
当模数 p 为质数时,a 的逆元即为 $a^{p-2}$,故直接用快速幂计算即可.
\begin{lstlisting}
qmi(a, p - 2, p);
\end{lstlisting}

\chapter{扩展欧几里得算法(粉碎机)}
\section{扩展欧几里得算法原理}
\begin{lstlisting}
int exgcd(int a, int b, int &x, int &y)
{
    if(!b)
    {
        x = 1, y = 0;
        return a;
    }
    
    int d = exgcd(b, a % b, y, x);
    y = y - a / b * x;
    return d;
}
\end{lstlisting}
\section{扩展欧几里得算法求逆元}
\begin{lstlisting}
int exgcd(int a, int p, int &x, int &y)
{
    if(!p)
    {
        x = 1, y = 0;
        return a;
    }
    
    int d = exgcd(p, a % p, y, x);
    y = y - a / p * x;
    return d;
}
\end{lstlisting}
运算结束后首先判断返回值 d 是否为 1, 因为 a 的逆元存在的条件是 $\gcd(a, p) = 1$,即两数互质,若 a 的逆元存在则
\[(x + p) \% p\]
为 a 的逆元
\chapter{中国剩余定理(孙子定理)}

\chapter{组合数 $C_a^b$}
\section{a、b较小的组合数}
\begin{lstlisting}
#include <cstdio>
using namespace std;
const int N = 2010, mod = 1e9 + 7;

int f[N][N];

int main()
{
    int n;
    scanf("%d", &n);
    
    for(int i = 0; i < N; i ++)
        for(int j = 0; j <= i; j ++)
        {
            if(!j) f[i][j] = 1;
            else f[i][j] = ((long long)f[i - 1][j] + f[i - 1][j - 1]) % mod;
        }
    
    while(n --)
    {
        int a, b;
        scanf("%d%d", &a, &b);
        printf("%d\n", f[a][b]);
    }
    return 0;
}
\end{lstlisting}
\[ \mathrm{C}_a^b = \mathrm{C}_{a-1}^{b-1} + \mathrm{C}_{a-1}^b\]

类似动态规划,从 $a$ 中选 $b$ 个的方法数量等于不选第b个物品的方法数量加上选第b个物品的方法数量。即从 $a - 1$ 中选 $b$ 的方法数量加上从
$a - 1$ 中选 $b - 1$ 的方法数量。
\section{a、b较大的组合数}
\begin{lstlisting}
#include <cstdio>
using namespace std;
typedef long long LL;
const int N = 1e5 + 10, mod = 1e9 + 7;

int infact[N], fact[N];
int T;

int qmi(int a, int b, int p)
{
    int res = 1;
    while(b)
    {
        if(b & 1) res = (LL)res * a % p;
        a = (LL)a * a % p;
        b >>= 1;
    }
    return res;
}

int main()
{
    scanf("%d", &T);
    
    fact[0] = 1, infact[0] = 1;
    for(int i = 1; i < N; i ++)
    {
        fact[i] = (LL)fact[i - 1] * i % mod;
        infact[i] = (LL)infact[i - 1] * qmi(i, mod - 2, mod) % mod;
    }
    
    while(T --)
    {
        int a, b;
        scanf("%d%d", &a, &b);
        printf("%d\n", (LL)fact[a] * infact[b] % mod * infact[a - b] % mod);
    }
    
    return 0;
}
\end{lstlisting}
\[ \mathrm{C}_a^b = \frac{a!}{b!(a-b)!}\]
预处理出从 1 到 N 每个数的阶乘及阶乘的逆元即可。除法可以转换为乘以逆元。
这里的模数 1e9+7 为质数, 所以求逆元时可以方便地直接用快速幂。
\section{a、b特别大的组合数}
\begin{lstlisting}
#include <cstdio>
using namespace std;
typedef long long LL;

int qmi(LL a, LL b, int p)
{
    int res = 1;
    while(b)
    {
        if(b & 1) res = (LL)res * a % p;
        a = a * a % p;
        b >>= 1;
    }
    return res;
}

int C(LL a, LL b, int p)
{
    LL res = 1;
    for(int i = 1, j = a; i <= b; i ++, j --)
    {
        res = (LL)res * j % p;
        res = (LL)res * qmi(i, p - 2, p) % p;
    }

    return res;
}

int lucas(LL a, LL b, int p)
{
    if(a < p && b <p) return C(a, b, p);
    else return (LL)lucas(a % p, b % p, p) * lucas(a / p, b / p, p) % p;
}

int main()
{
    int n;
    scanf("%d", &n);
    while(n --)
    {
        LL a, b;
        int p;
        scanf("%lld%lld%d", &a, &b, &p);
        printf("%d\n", lucas(a, b, p));
    }
    return 0;
}
\end{lstlisting}
\[ \mathrm{C}_a^b = \frac{a!}{b!(a-b)!} = \frac{a*(a-1)*(a-2)*\cdots*(a-b+2)*(a-b+1)}{b!}\]
卢卡斯定理
\[\mathrm{C}_a^b = \mathrm{C}_{a\bmod p}^{b\bmod p} + \mathrm{C}_{\frac{a}{p}}^{\frac{b}{p}} \pmod{p}\]

本题每次询问的 p 是不固定的,所以不能像上题预处理阶乘和阶乘的逆元后直接用来计算。代码中的$C$函数即是对组合数计算的模拟。

\section{不取模的高精度组合数}
%\verb|5 3|
\begin{lstlisting}
#include <cstdio>
#include <vector>
#include <cstring>
using namespace std;
const int N = 5010;

int primes[N], cnt, sum[N];
bool st[N];

void get_primes(int x)
{
    for(int i = 2; i <= x; i ++)
    {
        if(!st[i]) primes[cnt ++] = i;
        for(int j = 0; primes[j] <= x / i; j ++)
        {
            st[primes[j] * i] = true;
            if(i % primes[j] == 0) break;
        }
    }
}

int get_sum(int a, int p)
{
    int res = 0;
    while(a)
    {
        res += a / p;
        a /= p;
    }
    return res;
}

vector<int> mul(vector<int>a, int b)
{
    vector<int> res;
    int t = 0;
    for(int i = 0; i < a.size(); i ++)
    {
        t += a[i] * b;
        res.push_back(t % 10);
        t /= 10;
    }

    while(t)
    {
        res.push_back(t % 10);
        t /= 10;
    }
    return res;
}

int main()
{
    int a, b;
    scanf("%d%d", &a, &b);
    get_primes(a);

    //阶乘分解,记录每个质因数的次数
    for(int i = 0; i < cnt; i ++)
    {
        int p = primes[i];
        sum[i] = get_sum(a, p) - get_sum(b, p) - get_sum(a - b, p);
    }

    //高精度乘法
    vector<int> ans;
    ans.push_back(1);
    for(int i = 0; i < cnt; i ++)
        for(int j = 1; j <= sum[i]; j ++)
            ans = mul(ans, primes[i]);

    for(int i = ans.size() - 1; i >= 0; i --)
        printf("%d", ans[i]);

    puts("");
    return 0;
}    
\end{lstlisting}

对组合数计算中的阶乘做阶乘分解,即将阶乘分解为算术基本定理的形式。
\[n = p_{1}^{a_{1}} + p_{2}^{a_{2}} + p_{3}^{a_{3}} +\cdots+ p_{n}^{a_{n}}\]
$get\_sum(a, p)$就是质因子$p$在$a!$中出现的次数,$get\_sum(b, p)$就是质因子$p$在$b!$中出现的次数,因为是除以$b!$,所以要减去,类似地$get\_sum(a - b, p)$也是一样。

记录每个质因子应该乘的次数,然后用高精度乘法将每个质因子乘以相应的次数即可。
\chapter{容斥原理}

\chapter{博弈论}
\end{document}